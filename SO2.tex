\documentclass{article}
\usepackage{hyperref}
\usepackage[catalan]{babel}
\usepackage{graphicx}
\selectlanguage{catalan}

\graphicspath{ {./img} }

\hypersetup{
    colorlinks,
    citecolor=black,
    filecolor=black,
    linkcolor=black,
    urlcolor=blue
}

\title{Sistemes Operatius 2 (SO2)}
\date{2018-2019 Q2}
\author{David Soldevila}

\begin{document}

    \maketitle  

    \pagebreak
    
    \tableofcontents
    
    \pagebreak

    \section{Introducció}

    Treballarem amb x86 amb regs de 32 bits.

    Registres importants:
    \begin{itemize}
        \item EAX
        \item EBX
        \item ESI
        \item EDI
        \item ECX
        \item EBP
        \item EDX     
        \item ESP
    \end{itemize}

    \pagebreak

    \section{Mecanismes d'integritat del Sistema Operatiu}

    \subsection{Nivells de privilegi}

    Fa que els usuaris no tinguin accés al hardware. Les instruccions privilagiades només poden ser executades per l'OS. Es necessaria suport de hardware.

    \subsection{Vector d'interrupcions}

    IDT: Interrupt Descriptor Table: 256 entrades

    - 0 - 31: Exceptions
    - 32 - 47: Masked interrupts
    - 48 - 255: Software interrupts (Traps)

    El tractament de la excepcio s'hauria de fer en una sola rutina.

    \subsection{Excepcions hardware}

    S'ha de notificar al controlador de interrupcions quan s'ha acabat de tractar, per tal de poder tractar mes interrupcions.

\end{document}